\documentclass[a4paper, 12pt]{jsreport}
\input preemble.tex
\input title.tex

% prevent hyphenation
\hyphenpenalty=10000\relax
\exhyphenpenalty=10000\relax
\sloppy

\begin{document}
\maketitle

\question{2.}
\begin{qparts}
    \qpart
        \begin{qlist2}
            \qitem 略. 
            \qitem Rewrite the INSERTION-SORT procedure to sort into nonincreasing instead of non-decreasing order.
\begin{lstlisting}[caption=DESC-INSERTION-SORT(\textit{A}), label=2.1-2, language=C++]
for j = 2 to A.length
    key = A[j]
    i = j - 1
    while i > 0 && A[i] < key
        A[i+1] = A[i]
        i = i - 1
    A[i+1] = key
\end{lstlisting}
            \qitem Write pseudocode for \textit{\textbf{linear search}}. 
\begin{lstlisting}[caption=LINEAR-SEARCH(\textit{A} and $v$), label=2.1-3, language=C++]
for j = 2 to A.length
    key = A[j]
    i = j - 1
    while i > 0 && A[i] < key
        A[i+1] = A[i]
        i = i - 1
    A[i+1] = key
\end{lstlisting}
            \qitem Consider the problem of adding two $n$-bit binary integers, stored in two $n$-element arrays \textit{A} and \textit{B}. The sum of the two integers should be stored in binary form in an $(n+1)$-element array \textit{C}. State the problem formally and write pseudocode for adding the two integers. 
            \begin{lstlisting}[caption=BINARY-PLUS-OPERATOR(\textit{A} and \textit{B}), label=2.1-4, language=C++]
for j = 2 to A.length
    key = A[j]
    i = j - 1
    while i > 0 && A[i] < key
        A[i+1] = A[i]
        i = i - 1
    A[i+1] = key
\end{lstlisting}
        \end{qlist2}
    \qpart 次に、宇宙が10次元の場合を考えよう。
        \begin{qlist2}
            \qitem それを計算せよ。\label{q:10dim}
            \qitem \qref{q:10dim}の解を用いてあれを証明せよ。
        \end{qlist2}
\end{qparts}

\end{document}
